
% Default to the notebook output style

    


% Inherit from the specified cell style.




    
\documentclass[11pt]{article}

    
    
    \usepackage[T1]{fontenc}
    % Nicer default font (+ math font) than Computer Modern for most use cases
    \usepackage{mathpazo}

    % Basic figure setup, for now with no caption control since it's done
    % automatically by Pandoc (which extracts ![](path) syntax from Markdown).
    \usepackage{graphicx}
    % We will generate all images so they have a width \maxwidth. This means
    % that they will get their normal width if they fit onto the page, but
    % are scaled down if they would overflow the margins.
    \makeatletter
    \def\maxwidth{\ifdim\Gin@nat@width>\linewidth\linewidth
    \else\Gin@nat@width\fi}
    \makeatother
    \let\Oldincludegraphics\includegraphics
    % Set max figure width to be 80% of text width, for now hardcoded.
    \renewcommand{\includegraphics}[1]{\Oldincludegraphics[width=.8\maxwidth]{#1}}
    % Ensure that by default, figures have no caption (until we provide a
    % proper Figure object with a Caption API and a way to capture that
    % in the conversion process - todo).
    \usepackage{caption}
    \DeclareCaptionLabelFormat{nolabel}{}
    \captionsetup{labelformat=nolabel}

    \usepackage{adjustbox} % Used to constrain images to a maximum size 
    \usepackage{xcolor} % Allow colors to be defined
    \usepackage{enumerate} % Needed for markdown enumerations to work
    \usepackage{geometry} % Used to adjust the document margins
    \usepackage{amsmath} % Equations
    \usepackage{amssymb} % Equations
    \usepackage{textcomp} % defines textquotesingle
    % Hack from http://tex.stackexchange.com/a/47451/13684:
    \AtBeginDocument{%
        \def\PYZsq{\textquotesingle}% Upright quotes in Pygmentized code
    }
    \usepackage{upquote} % Upright quotes for verbatim code
    \usepackage{eurosym} % defines \euro
    \usepackage[mathletters]{ucs} % Extended unicode (utf-8) support
    \usepackage[utf8x]{inputenc} % Allow utf-8 characters in the tex document
    \usepackage{fancyvrb} % verbatim replacement that allows latex
    \usepackage{grffile} % extends the file name processing of package graphics 
                         % to support a larger range 
    % The hyperref package gives us a pdf with properly built
    % internal navigation ('pdf bookmarks' for the table of contents,
    % internal cross-reference links, web links for URLs, etc.)
    \usepackage{hyperref}
    \usepackage{longtable} % longtable support required by pandoc >1.10
    \usepackage{booktabs}  % table support for pandoc > 1.12.2
    \usepackage[inline]{enumitem} % IRkernel/repr support (it uses the enumerate* environment)
    \usepackage[normalem]{ulem} % ulem is needed to support strikethroughs (\sout)
                                % normalem makes italics be italics, not underlines
    

    
    
    % Colors for the hyperref package
    \definecolor{urlcolor}{rgb}{0,.145,.698}
    \definecolor{linkcolor}{rgb}{.71,0.21,0.01}
    \definecolor{citecolor}{rgb}{.12,.54,.11}

    % ANSI colors
    \definecolor{ansi-black}{HTML}{3E424D}
    \definecolor{ansi-black-intense}{HTML}{282C36}
    \definecolor{ansi-red}{HTML}{E75C58}
    \definecolor{ansi-red-intense}{HTML}{B22B31}
    \definecolor{ansi-green}{HTML}{00A250}
    \definecolor{ansi-green-intense}{HTML}{007427}
    \definecolor{ansi-yellow}{HTML}{DDB62B}
    \definecolor{ansi-yellow-intense}{HTML}{B27D12}
    \definecolor{ansi-blue}{HTML}{208FFB}
    \definecolor{ansi-blue-intense}{HTML}{0065CA}
    \definecolor{ansi-magenta}{HTML}{D160C4}
    \definecolor{ansi-magenta-intense}{HTML}{A03196}
    \definecolor{ansi-cyan}{HTML}{60C6C8}
    \definecolor{ansi-cyan-intense}{HTML}{258F8F}
    \definecolor{ansi-white}{HTML}{C5C1B4}
    \definecolor{ansi-white-intense}{HTML}{A1A6B2}

    % commands and environments needed by pandoc snippets
    % extracted from the output of `pandoc -s`
    \providecommand{\tightlist}{%
      \setlength{\itemsep}{0pt}\setlength{\parskip}{0pt}}
    \DefineVerbatimEnvironment{Highlighting}{Verbatim}{commandchars=\\\{\}}
    % Add ',fontsize=\small' for more characters per line
    \newenvironment{Shaded}{}{}
    \newcommand{\KeywordTok}[1]{\textcolor[rgb]{0.00,0.44,0.13}{\textbf{{#1}}}}
    \newcommand{\DataTypeTok}[1]{\textcolor[rgb]{0.56,0.13,0.00}{{#1}}}
    \newcommand{\DecValTok}[1]{\textcolor[rgb]{0.25,0.63,0.44}{{#1}}}
    \newcommand{\BaseNTok}[1]{\textcolor[rgb]{0.25,0.63,0.44}{{#1}}}
    \newcommand{\FloatTok}[1]{\textcolor[rgb]{0.25,0.63,0.44}{{#1}}}
    \newcommand{\CharTok}[1]{\textcolor[rgb]{0.25,0.44,0.63}{{#1}}}
    \newcommand{\StringTok}[1]{\textcolor[rgb]{0.25,0.44,0.63}{{#1}}}
    \newcommand{\CommentTok}[1]{\textcolor[rgb]{0.38,0.63,0.69}{\textit{{#1}}}}
    \newcommand{\OtherTok}[1]{\textcolor[rgb]{0.00,0.44,0.13}{{#1}}}
    \newcommand{\AlertTok}[1]{\textcolor[rgb]{1.00,0.00,0.00}{\textbf{{#1}}}}
    \newcommand{\FunctionTok}[1]{\textcolor[rgb]{0.02,0.16,0.49}{{#1}}}
    \newcommand{\RegionMarkerTok}[1]{{#1}}
    \newcommand{\ErrorTok}[1]{\textcolor[rgb]{1.00,0.00,0.00}{\textbf{{#1}}}}
    \newcommand{\NormalTok}[1]{{#1}}
    
    % Additional commands for more recent versions of Pandoc
    \newcommand{\ConstantTok}[1]{\textcolor[rgb]{0.53,0.00,0.00}{{#1}}}
    \newcommand{\SpecialCharTok}[1]{\textcolor[rgb]{0.25,0.44,0.63}{{#1}}}
    \newcommand{\VerbatimStringTok}[1]{\textcolor[rgb]{0.25,0.44,0.63}{{#1}}}
    \newcommand{\SpecialStringTok}[1]{\textcolor[rgb]{0.73,0.40,0.53}{{#1}}}
    \newcommand{\ImportTok}[1]{{#1}}
    \newcommand{\DocumentationTok}[1]{\textcolor[rgb]{0.73,0.13,0.13}{\textit{{#1}}}}
    \newcommand{\AnnotationTok}[1]{\textcolor[rgb]{0.38,0.63,0.69}{\textbf{\textit{{#1}}}}}
    \newcommand{\CommentVarTok}[1]{\textcolor[rgb]{0.38,0.63,0.69}{\textbf{\textit{{#1}}}}}
    \newcommand{\VariableTok}[1]{\textcolor[rgb]{0.10,0.09,0.49}{{#1}}}
    \newcommand{\ControlFlowTok}[1]{\textcolor[rgb]{0.00,0.44,0.13}{\textbf{{#1}}}}
    \newcommand{\OperatorTok}[1]{\textcolor[rgb]{0.40,0.40,0.40}{{#1}}}
    \newcommand{\BuiltInTok}[1]{{#1}}
    \newcommand{\ExtensionTok}[1]{{#1}}
    \newcommand{\PreprocessorTok}[1]{\textcolor[rgb]{0.74,0.48,0.00}{{#1}}}
    \newcommand{\AttributeTok}[1]{\textcolor[rgb]{0.49,0.56,0.16}{{#1}}}
    \newcommand{\InformationTok}[1]{\textcolor[rgb]{0.38,0.63,0.69}{\textbf{\textit{{#1}}}}}
    \newcommand{\WarningTok}[1]{\textcolor[rgb]{0.38,0.63,0.69}{\textbf{\textit{{#1}}}}}
    
    
    % Define a nice break command that doesn't care if a line doesn't already
    % exist.
    \def\br{\hspace*{\fill} \\* }
    % Math Jax compatability definitions
    \def\gt{>}
    \def\lt{<}
    % Document parameters
    \title{model\_training}
    
    
    

    % Pygments definitions
    
\makeatletter
\def\PY@reset{\let\PY@it=\relax \let\PY@bf=\relax%
    \let\PY@ul=\relax \let\PY@tc=\relax%
    \let\PY@bc=\relax \let\PY@ff=\relax}
\def\PY@tok#1{\csname PY@tok@#1\endcsname}
\def\PY@toks#1+{\ifx\relax#1\empty\else%
    \PY@tok{#1}\expandafter\PY@toks\fi}
\def\PY@do#1{\PY@bc{\PY@tc{\PY@ul{%
    \PY@it{\PY@bf{\PY@ff{#1}}}}}}}
\def\PY#1#2{\PY@reset\PY@toks#1+\relax+\PY@do{#2}}

\expandafter\def\csname PY@tok@cpf\endcsname{\let\PY@it=\textit\def\PY@tc##1{\textcolor[rgb]{0.25,0.50,0.50}{##1}}}
\expandafter\def\csname PY@tok@ne\endcsname{\let\PY@bf=\textbf\def\PY@tc##1{\textcolor[rgb]{0.82,0.25,0.23}{##1}}}
\expandafter\def\csname PY@tok@nv\endcsname{\def\PY@tc##1{\textcolor[rgb]{0.10,0.09,0.49}{##1}}}
\expandafter\def\csname PY@tok@mb\endcsname{\def\PY@tc##1{\textcolor[rgb]{0.40,0.40,0.40}{##1}}}
\expandafter\def\csname PY@tok@ow\endcsname{\let\PY@bf=\textbf\def\PY@tc##1{\textcolor[rgb]{0.67,0.13,1.00}{##1}}}
\expandafter\def\csname PY@tok@dl\endcsname{\def\PY@tc##1{\textcolor[rgb]{0.73,0.13,0.13}{##1}}}
\expandafter\def\csname PY@tok@cp\endcsname{\def\PY@tc##1{\textcolor[rgb]{0.74,0.48,0.00}{##1}}}
\expandafter\def\csname PY@tok@o\endcsname{\def\PY@tc##1{\textcolor[rgb]{0.40,0.40,0.40}{##1}}}
\expandafter\def\csname PY@tok@gh\endcsname{\let\PY@bf=\textbf\def\PY@tc##1{\textcolor[rgb]{0.00,0.00,0.50}{##1}}}
\expandafter\def\csname PY@tok@gr\endcsname{\def\PY@tc##1{\textcolor[rgb]{1.00,0.00,0.00}{##1}}}
\expandafter\def\csname PY@tok@sb\endcsname{\def\PY@tc##1{\textcolor[rgb]{0.73,0.13,0.13}{##1}}}
\expandafter\def\csname PY@tok@vm\endcsname{\def\PY@tc##1{\textcolor[rgb]{0.10,0.09,0.49}{##1}}}
\expandafter\def\csname PY@tok@gi\endcsname{\def\PY@tc##1{\textcolor[rgb]{0.00,0.63,0.00}{##1}}}
\expandafter\def\csname PY@tok@s1\endcsname{\def\PY@tc##1{\textcolor[rgb]{0.73,0.13,0.13}{##1}}}
\expandafter\def\csname PY@tok@kn\endcsname{\let\PY@bf=\textbf\def\PY@tc##1{\textcolor[rgb]{0.00,0.50,0.00}{##1}}}
\expandafter\def\csname PY@tok@c\endcsname{\let\PY@it=\textit\def\PY@tc##1{\textcolor[rgb]{0.25,0.50,0.50}{##1}}}
\expandafter\def\csname PY@tok@gs\endcsname{\let\PY@bf=\textbf}
\expandafter\def\csname PY@tok@mf\endcsname{\def\PY@tc##1{\textcolor[rgb]{0.40,0.40,0.40}{##1}}}
\expandafter\def\csname PY@tok@vg\endcsname{\def\PY@tc##1{\textcolor[rgb]{0.10,0.09,0.49}{##1}}}
\expandafter\def\csname PY@tok@vi\endcsname{\def\PY@tc##1{\textcolor[rgb]{0.10,0.09,0.49}{##1}}}
\expandafter\def\csname PY@tok@cm\endcsname{\let\PY@it=\textit\def\PY@tc##1{\textcolor[rgb]{0.25,0.50,0.50}{##1}}}
\expandafter\def\csname PY@tok@gu\endcsname{\let\PY@bf=\textbf\def\PY@tc##1{\textcolor[rgb]{0.50,0.00,0.50}{##1}}}
\expandafter\def\csname PY@tok@kr\endcsname{\let\PY@bf=\textbf\def\PY@tc##1{\textcolor[rgb]{0.00,0.50,0.00}{##1}}}
\expandafter\def\csname PY@tok@nt\endcsname{\let\PY@bf=\textbf\def\PY@tc##1{\textcolor[rgb]{0.00,0.50,0.00}{##1}}}
\expandafter\def\csname PY@tok@sr\endcsname{\def\PY@tc##1{\textcolor[rgb]{0.73,0.40,0.53}{##1}}}
\expandafter\def\csname PY@tok@nd\endcsname{\def\PY@tc##1{\textcolor[rgb]{0.67,0.13,1.00}{##1}}}
\expandafter\def\csname PY@tok@go\endcsname{\def\PY@tc##1{\textcolor[rgb]{0.53,0.53,0.53}{##1}}}
\expandafter\def\csname PY@tok@s\endcsname{\def\PY@tc##1{\textcolor[rgb]{0.73,0.13,0.13}{##1}}}
\expandafter\def\csname PY@tok@na\endcsname{\def\PY@tc##1{\textcolor[rgb]{0.49,0.56,0.16}{##1}}}
\expandafter\def\csname PY@tok@s2\endcsname{\def\PY@tc##1{\textcolor[rgb]{0.73,0.13,0.13}{##1}}}
\expandafter\def\csname PY@tok@ch\endcsname{\let\PY@it=\textit\def\PY@tc##1{\textcolor[rgb]{0.25,0.50,0.50}{##1}}}
\expandafter\def\csname PY@tok@c1\endcsname{\let\PY@it=\textit\def\PY@tc##1{\textcolor[rgb]{0.25,0.50,0.50}{##1}}}
\expandafter\def\csname PY@tok@kt\endcsname{\def\PY@tc##1{\textcolor[rgb]{0.69,0.00,0.25}{##1}}}
\expandafter\def\csname PY@tok@sa\endcsname{\def\PY@tc##1{\textcolor[rgb]{0.73,0.13,0.13}{##1}}}
\expandafter\def\csname PY@tok@cs\endcsname{\let\PY@it=\textit\def\PY@tc##1{\textcolor[rgb]{0.25,0.50,0.50}{##1}}}
\expandafter\def\csname PY@tok@gp\endcsname{\let\PY@bf=\textbf\def\PY@tc##1{\textcolor[rgb]{0.00,0.00,0.50}{##1}}}
\expandafter\def\csname PY@tok@nb\endcsname{\def\PY@tc##1{\textcolor[rgb]{0.00,0.50,0.00}{##1}}}
\expandafter\def\csname PY@tok@vc\endcsname{\def\PY@tc##1{\textcolor[rgb]{0.10,0.09,0.49}{##1}}}
\expandafter\def\csname PY@tok@si\endcsname{\let\PY@bf=\textbf\def\PY@tc##1{\textcolor[rgb]{0.73,0.40,0.53}{##1}}}
\expandafter\def\csname PY@tok@sh\endcsname{\def\PY@tc##1{\textcolor[rgb]{0.73,0.13,0.13}{##1}}}
\expandafter\def\csname PY@tok@sx\endcsname{\def\PY@tc##1{\textcolor[rgb]{0.00,0.50,0.00}{##1}}}
\expandafter\def\csname PY@tok@no\endcsname{\def\PY@tc##1{\textcolor[rgb]{0.53,0.00,0.00}{##1}}}
\expandafter\def\csname PY@tok@nn\endcsname{\let\PY@bf=\textbf\def\PY@tc##1{\textcolor[rgb]{0.00,0.00,1.00}{##1}}}
\expandafter\def\csname PY@tok@kp\endcsname{\def\PY@tc##1{\textcolor[rgb]{0.00,0.50,0.00}{##1}}}
\expandafter\def\csname PY@tok@ni\endcsname{\let\PY@bf=\textbf\def\PY@tc##1{\textcolor[rgb]{0.60,0.60,0.60}{##1}}}
\expandafter\def\csname PY@tok@k\endcsname{\let\PY@bf=\textbf\def\PY@tc##1{\textcolor[rgb]{0.00,0.50,0.00}{##1}}}
\expandafter\def\csname PY@tok@bp\endcsname{\def\PY@tc##1{\textcolor[rgb]{0.00,0.50,0.00}{##1}}}
\expandafter\def\csname PY@tok@nf\endcsname{\def\PY@tc##1{\textcolor[rgb]{0.00,0.00,1.00}{##1}}}
\expandafter\def\csname PY@tok@gd\endcsname{\def\PY@tc##1{\textcolor[rgb]{0.63,0.00,0.00}{##1}}}
\expandafter\def\csname PY@tok@fm\endcsname{\def\PY@tc##1{\textcolor[rgb]{0.00,0.00,1.00}{##1}}}
\expandafter\def\csname PY@tok@nl\endcsname{\def\PY@tc##1{\textcolor[rgb]{0.63,0.63,0.00}{##1}}}
\expandafter\def\csname PY@tok@ge\endcsname{\let\PY@it=\textit}
\expandafter\def\csname PY@tok@se\endcsname{\let\PY@bf=\textbf\def\PY@tc##1{\textcolor[rgb]{0.73,0.40,0.13}{##1}}}
\expandafter\def\csname PY@tok@kc\endcsname{\let\PY@bf=\textbf\def\PY@tc##1{\textcolor[rgb]{0.00,0.50,0.00}{##1}}}
\expandafter\def\csname PY@tok@ss\endcsname{\def\PY@tc##1{\textcolor[rgb]{0.10,0.09,0.49}{##1}}}
\expandafter\def\csname PY@tok@w\endcsname{\def\PY@tc##1{\textcolor[rgb]{0.73,0.73,0.73}{##1}}}
\expandafter\def\csname PY@tok@gt\endcsname{\def\PY@tc##1{\textcolor[rgb]{0.00,0.27,0.87}{##1}}}
\expandafter\def\csname PY@tok@kd\endcsname{\let\PY@bf=\textbf\def\PY@tc##1{\textcolor[rgb]{0.00,0.50,0.00}{##1}}}
\expandafter\def\csname PY@tok@nc\endcsname{\let\PY@bf=\textbf\def\PY@tc##1{\textcolor[rgb]{0.00,0.00,1.00}{##1}}}
\expandafter\def\csname PY@tok@sd\endcsname{\let\PY@it=\textit\def\PY@tc##1{\textcolor[rgb]{0.73,0.13,0.13}{##1}}}
\expandafter\def\csname PY@tok@mo\endcsname{\def\PY@tc##1{\textcolor[rgb]{0.40,0.40,0.40}{##1}}}
\expandafter\def\csname PY@tok@mi\endcsname{\def\PY@tc##1{\textcolor[rgb]{0.40,0.40,0.40}{##1}}}
\expandafter\def\csname PY@tok@il\endcsname{\def\PY@tc##1{\textcolor[rgb]{0.40,0.40,0.40}{##1}}}
\expandafter\def\csname PY@tok@m\endcsname{\def\PY@tc##1{\textcolor[rgb]{0.40,0.40,0.40}{##1}}}
\expandafter\def\csname PY@tok@err\endcsname{\def\PY@bc##1{\setlength{\fboxsep}{0pt}\fcolorbox[rgb]{1.00,0.00,0.00}{1,1,1}{\strut ##1}}}
\expandafter\def\csname PY@tok@sc\endcsname{\def\PY@tc##1{\textcolor[rgb]{0.73,0.13,0.13}{##1}}}
\expandafter\def\csname PY@tok@mh\endcsname{\def\PY@tc##1{\textcolor[rgb]{0.40,0.40,0.40}{##1}}}

\def\PYZbs{\char`\\}
\def\PYZus{\char`\_}
\def\PYZob{\char`\{}
\def\PYZcb{\char`\}}
\def\PYZca{\char`\^}
\def\PYZam{\char`\&}
\def\PYZlt{\char`\<}
\def\PYZgt{\char`\>}
\def\PYZsh{\char`\#}
\def\PYZpc{\char`\%}
\def\PYZdl{\char`\$}
\def\PYZhy{\char`\-}
\def\PYZsq{\char`\'}
\def\PYZdq{\char`\"}
\def\PYZti{\char`\~}
% for compatibility with earlier versions
\def\PYZat{@}
\def\PYZlb{[}
\def\PYZrb{]}
\makeatother


    % Exact colors from NB
    \definecolor{incolor}{rgb}{0.0, 0.0, 0.5}
    \definecolor{outcolor}{rgb}{0.545, 0.0, 0.0}



    
    % Prevent overflowing lines due to hard-to-break entities
    \sloppy 
    % Setup hyperref package
    \hypersetup{
      breaklinks=true,  % so long urls are correctly broken across lines
      colorlinks=true,
      urlcolor=urlcolor,
      linkcolor=linkcolor,
      citecolor=citecolor,
      }
    % Slightly bigger margins than the latex defaults
    
    \geometry{verbose,tmargin=1in,bmargin=1in,lmargin=1in,rmargin=1in}
    
    

    \begin{document}
    
    
    \maketitle
    
    

    
    \hypertarget{follow-me-project}{%
\section{Follow-Me Project}\label{follow-me-project}}

Congratulations on reaching the final project of the Robotics
Nanodegree!

Previously, you worked on the Semantic Segmentation lab where you built
a deep learning network that locates a particular human target within an
image. For this project, you will utilize what you implemented and
learned from that lab and extend it to train a deep learning model that
will allow a simulated quadcopter to follow around the person that it
detects!

Most of the code below is similar to the lab with some minor
modifications. You can start with your existing solution, and modify and
improve upon it to train the best possible model for this task.

You can click on any of the following to quickly jump to that part of
this notebook: 1. Section \ref{data} 2. Section \ref{fcn} 3.
Section \ref{build} 4. Section \ref{training} 5.
Section \ref{prediction} 6. Section \ref{evaluation}

    \hypertarget{data-collection}{%
\subsection{Data Collection}\label{data-collection}}

We have provided you with a starting dataset for this project. Download
instructions can be found in the README for this project's repo.
Alternatively, you can collect additional data of your own to improve
your model. Check out the ``Collecting Data'' section in the Project
Lesson in the Classroom for more details!

    \begin{Verbatim}[commandchars=\\\{\}]
{\color{incolor}In [{\color{incolor}1}]:} \PY{k+kn}{import} \PY{n+nn}{os}
        \PY{k+kn}{import} \PY{n+nn}{glob}
        \PY{k+kn}{import} \PY{n+nn}{sys}
        \PY{k+kn}{import} \PY{n+nn}{tensorflow} \PY{k}{as} \PY{n+nn}{tf}
        
        \PY{k+kn}{from} \PY{n+nn}{scipy} \PY{k}{import} \PY{n}{misc}
        \PY{k+kn}{import} \PY{n+nn}{numpy} \PY{k}{as} \PY{n+nn}{np}
        
        \PY{k+kn}{from} \PY{n+nn}{tensorflow}\PY{n+nn}{.}\PY{n+nn}{contrib}\PY{n+nn}{.}\PY{n+nn}{keras}\PY{n+nn}{.}\PY{n+nn}{python} \PY{k}{import} \PY{n}{keras}
        \PY{k+kn}{from} \PY{n+nn}{tensorflow}\PY{n+nn}{.}\PY{n+nn}{contrib}\PY{n+nn}{.}\PY{n+nn}{keras}\PY{n+nn}{.}\PY{n+nn}{python}\PY{n+nn}{.}\PY{n+nn}{keras} \PY{k}{import} \PY{n}{layers}\PY{p}{,} \PY{n}{models}
        
        \PY{k+kn}{from} \PY{n+nn}{tensorflow} \PY{k}{import} \PY{n}{image}
        
        \PY{k+kn}{from} \PY{n+nn}{utils} \PY{k}{import} \PY{n}{scoring\PYZus{}utils}
        \PY{k+kn}{from} \PY{n+nn}{utils}\PY{n+nn}{.}\PY{n+nn}{separable\PYZus{}conv2d} \PY{k}{import} \PY{n}{SeparableConv2DKeras}\PY{p}{,} \PY{n}{BilinearUpSampling2D}
        \PY{k+kn}{from} \PY{n+nn}{utils} \PY{k}{import} \PY{n}{data\PYZus{}iterator}
        \PY{k+kn}{from} \PY{n+nn}{utils} \PY{k}{import} \PY{n}{plotting\PYZus{}tools} 
        \PY{k+kn}{from} \PY{n+nn}{utils} \PY{k}{import} \PY{n}{model\PYZus{}tools}
\end{Verbatim}


    \hypertarget{fcn-layers}{%
\subsection{FCN Layers }\label{fcn-layers}}

In the Classroom, we discussed the different layers that constitute a
fully convolutional network (FCN). The following code will introduce you
to the functions that you need to build your semantic segmentation
model.

    \hypertarget{separable-convolutions}{%
\subsubsection{Separable Convolutions}\label{separable-convolutions}}

The Encoder for your FCN will essentially require separable convolution
layers, due to their advantages as explained in the classroom. The 1x1
convolution layer in the FCN, however, is a regular convolution.
Implementations for both are provided below for your use. Each includes
batch normalization with the ReLU activation function applied to the
layers.

    \begin{Verbatim}[commandchars=\\\{\}]
{\color{incolor}In [{\color{incolor}2}]:} \PY{k}{def} \PY{n+nf}{separable\PYZus{}conv2d\PYZus{}batchnorm}\PY{p}{(}\PY{n}{input\PYZus{}layer}\PY{p}{,} \PY{n}{filters}\PY{p}{,} \PY{n}{strides}\PY{o}{=}\PY{l+m+mi}{1}\PY{p}{)}\PY{p}{:}
            \PY{n}{output\PYZus{}layer} \PY{o}{=} \PY{n}{SeparableConv2DKeras}\PY{p}{(}\PY{n}{filters}\PY{o}{=}\PY{n}{filters}\PY{p}{,}\PY{n}{kernel\PYZus{}size}\PY{o}{=}\PY{l+m+mi}{3}\PY{p}{,} \PY{n}{strides}\PY{o}{=}\PY{n}{strides}\PY{p}{,}
                                     \PY{n}{padding}\PY{o}{=}\PY{l+s+s1}{\PYZsq{}}\PY{l+s+s1}{same}\PY{l+s+s1}{\PYZsq{}}\PY{p}{,} \PY{n}{activation}\PY{o}{=}\PY{l+s+s1}{\PYZsq{}}\PY{l+s+s1}{relu}\PY{l+s+s1}{\PYZsq{}}\PY{p}{)}\PY{p}{(}\PY{n}{input\PYZus{}layer}\PY{p}{)}
            
            \PY{n}{output\PYZus{}layer} \PY{o}{=} \PY{n}{layers}\PY{o}{.}\PY{n}{BatchNormalization}\PY{p}{(}\PY{p}{)}\PY{p}{(}\PY{n}{output\PYZus{}layer}\PY{p}{)} 
            \PY{k}{return} \PY{n}{output\PYZus{}layer}
        
        \PY{k}{def} \PY{n+nf}{conv2d\PYZus{}batchnorm}\PY{p}{(}\PY{n}{input\PYZus{}layer}\PY{p}{,} \PY{n}{filters}\PY{p}{,} \PY{n}{kernel\PYZus{}size}\PY{o}{=}\PY{l+m+mi}{3}\PY{p}{,} \PY{n}{strides}\PY{o}{=}\PY{l+m+mi}{1}\PY{p}{)}\PY{p}{:}
            \PY{n}{output\PYZus{}layer} \PY{o}{=} \PY{n}{layers}\PY{o}{.}\PY{n}{Conv2D}\PY{p}{(}\PY{n}{filters}\PY{o}{=}\PY{n}{filters}\PY{p}{,} \PY{n}{kernel\PYZus{}size}\PY{o}{=}\PY{n}{kernel\PYZus{}size}\PY{p}{,} \PY{n}{strides}\PY{o}{=}\PY{n}{strides}\PY{p}{,} 
                              \PY{n}{padding}\PY{o}{=}\PY{l+s+s1}{\PYZsq{}}\PY{l+s+s1}{same}\PY{l+s+s1}{\PYZsq{}}\PY{p}{,} \PY{n}{activation}\PY{o}{=}\PY{l+s+s1}{\PYZsq{}}\PY{l+s+s1}{relu}\PY{l+s+s1}{\PYZsq{}}\PY{p}{)}\PY{p}{(}\PY{n}{input\PYZus{}layer}\PY{p}{)}
            
            \PY{n}{output\PYZus{}layer} \PY{o}{=} \PY{n}{layers}\PY{o}{.}\PY{n}{BatchNormalization}\PY{p}{(}\PY{p}{)}\PY{p}{(}\PY{n}{output\PYZus{}layer}\PY{p}{)} 
            \PY{k}{return} \PY{n}{output\PYZus{}layer}
\end{Verbatim}


    \hypertarget{bilinear-upsampling}{%
\subsubsection{Bilinear Upsampling}\label{bilinear-upsampling}}

The following helper function implements the bilinear upsampling layer.
Upsampling by a factor of 2 is generally recommended, but you can try
out different factors as well. Upsampling is used in the decoder block
of the FCN.

    \begin{Verbatim}[commandchars=\\\{\}]
{\color{incolor}In [{\color{incolor}3}]:} \PY{k}{def} \PY{n+nf}{bilinear\PYZus{}upsample}\PY{p}{(}\PY{n}{input\PYZus{}layer}\PY{p}{)}\PY{p}{:}
            \PY{n}{output\PYZus{}layer} \PY{o}{=} \PY{n}{BilinearUpSampling2D}\PY{p}{(}\PY{p}{(}\PY{l+m+mi}{2}\PY{p}{,}\PY{l+m+mi}{2}\PY{p}{)}\PY{p}{)}\PY{p}{(}\PY{n}{input\PYZus{}layer}\PY{p}{)}
            \PY{k}{return} \PY{n}{output\PYZus{}layer}
\end{Verbatim}


    \hypertarget{build-the-model}{%
\subsection{Build the Model }\label{build-the-model}}

In the following cells, you will build an FCN to train a model to detect
and locate the hero target within an image. The steps are: - Create an
\texttt{encoder\_block} - Create a \texttt{decoder\_block} - Build the
FCN consisting of encoder block(s), a 1x1 convolution, and decoder
block(s). This step requires experimentation with different numbers of
layers and filter sizes to build your model.

    \hypertarget{encoder-block}{%
\subsubsection{Encoder Block}\label{encoder-block}}

Create an encoder block that includes a separable convolution layer
using the \texttt{separable\_conv2d\_batchnorm()} function. The
\texttt{filters} parameter defines the size or depth of the output
layer. For example, 32 or 64.

    \begin{Verbatim}[commandchars=\\\{\}]
{\color{incolor}In [{\color{incolor}32}]:} \PY{k}{def} \PY{n+nf}{encoder\PYZus{}block}\PY{p}{(}\PY{n}{input\PYZus{}layer}\PY{p}{,} \PY{n}{filters}\PY{p}{,} \PY{n}{strides}\PY{p}{)}\PY{p}{:}
             
             \PY{c+c1}{\PYZsh{} TODO Create a separable convolution layer using the separable\PYZus{}conv2d\PYZus{}batchnorm() function.}
             \PY{n}{output\PYZus{}layer} \PY{o}{=} \PY{n}{separable\PYZus{}conv2d\PYZus{}batchnorm}\PY{p}{(}\PY{n}{input\PYZus{}layer}\PY{p}{,} \PY{n}{filters}\PY{p}{,} \PY{n}{strides}\PY{p}{)}
             \PY{k}{return} \PY{n}{output\PYZus{}layer}
\end{Verbatim}


    \hypertarget{decoder-block}{%
\subsubsection{Decoder Block}\label{decoder-block}}

The decoder block is comprised of three parts: - A bilinear upsampling
layer using the upsample\_bilinear() function. The current recommended
factor for upsampling is set to 2. - A layer concatenation step. This
step is similar to skip connections. You will concatenate the upsampled
small\_ip\_layer and the large\_ip\_layer. - Some (one or two)
additional separable convolution layers to extract some more spatial
information from prior layers.

    \begin{Verbatim}[commandchars=\\\{\}]
{\color{incolor}In [{\color{incolor}5}]:} \PY{k}{def} \PY{n+nf}{decoder\PYZus{}block}\PY{p}{(}\PY{n}{small\PYZus{}ip\PYZus{}layer}\PY{p}{,} \PY{n}{large\PYZus{}ip\PYZus{}layer}\PY{p}{,} \PY{n}{filters}\PY{p}{)}\PY{p}{:}
            
            \PY{c+c1}{\PYZsh{} TODO Upsample the small input layer using the bilinear\PYZus{}upsample() function.}
            \PY{n}{upsampled\PYZus{}layer} \PY{o}{=} \PY{n}{bilinear\PYZus{}upsample}\PY{p}{(}\PY{n}{small\PYZus{}ip\PYZus{}layer}\PY{p}{)}
            \PY{c+c1}{\PYZsh{} TODO Concatenate the upsampled and large input layers using layers.concatenate}
            \PY{n}{concatenated\PYZus{}layer} \PY{o}{=} \PY{n}{layers}\PY{o}{.}\PY{n}{concatenate}\PY{p}{(}\PY{p}{[}\PY{n}{upsampled\PYZus{}layer}\PY{p}{,} \PY{n}{large\PYZus{}ip\PYZus{}layer}\PY{p}{]}\PY{p}{)}
            \PY{c+c1}{\PYZsh{} TODO Add some number of separable convolution layers}
            \PY{n}{output\PYZus{}layer} \PY{o}{=} \PY{n}{separable\PYZus{}conv2d\PYZus{}batchnorm}\PY{p}{(}\PY{n}{concatenated\PYZus{}layer}\PY{p}{,} \PY{n}{filters}\PY{p}{)}
            \PY{k}{return} \PY{n}{output\PYZus{}layer}
\end{Verbatim}


    \hypertarget{model}{%
\subsubsection{Model}\label{model}}

Now that you have the encoder and decoder blocks ready, go ahead and
build your FCN architecture!

There are three steps: - Add encoder blocks to build the encoder layers.
This is similar to how you added regular convolutional layers in your
CNN lab. - Add a 1x1 Convolution layer using the conv2d\_batchnorm()
function. Remember that 1x1 Convolutions require a kernel and stride of
1. - Add decoder blocks for the decoder layers.

    \begin{Verbatim}[commandchars=\\\{\}]
{\color{incolor}In [{\color{incolor}6}]:} \PY{k}{def} \PY{n+nf}{fcn\PYZus{}model}\PY{p}{(}\PY{n}{inputs}\PY{p}{,} \PY{n}{num\PYZus{}classes}\PY{p}{)}\PY{p}{:}
            
            \PY{c+c1}{\PYZsh{} TODO Add Encoder Blocks. }
            \PY{c+c1}{\PYZsh{} Remember that with each encoder layer, the depth of your model (the number of filters) increases.}
            \PY{n}{enc\PYZus{}layer\PYZus{}1} \PY{o}{=} \PY{n}{encoder\PYZus{}block}\PY{p}{(}\PY{n}{inputs}\PY{p}{,} \PY{l+m+mi}{64}\PY{p}{,} \PY{l+m+mi}{2}\PY{p}{)}
            \PY{n}{enc\PYZus{}layer\PYZus{}2} \PY{o}{=} \PY{n}{encoder\PYZus{}block}\PY{p}{(}\PY{n}{enc\PYZus{}layer\PYZus{}1}\PY{p}{,} \PY{l+m+mi}{128}\PY{p}{,} \PY{l+m+mi}{2}\PY{p}{)}
            
            \PY{c+c1}{\PYZsh{} TODO Add 1x1 Convolution layer using conv2d\PYZus{}batchnorm().}
            \PY{n}{pixel\PYZus{}layer} \PY{o}{=} \PY{n}{conv2d\PYZus{}batchnorm}\PY{p}{(}\PY{n}{enc\PYZus{}layer\PYZus{}2}\PY{p}{,} \PY{l+m+mi}{256}\PY{p}{,} \PY{n}{kernel\PYZus{}size}\PY{o}{=}\PY{l+m+mi}{1}\PY{p}{,} \PY{n}{strides}\PY{o}{=}\PY{l+m+mi}{1}\PY{p}{)}
            
            \PY{c+c1}{\PYZsh{} TODO: Add the same number of Decoder Blocks as the number of Encoder Blocks}
            \PY{n}{dec\PYZus{}layer\PYZus{}1} \PY{o}{=} \PY{n}{decoder\PYZus{}block}\PY{p}{(}\PY{n}{pixel\PYZus{}layer}\PY{p}{,} \PY{n}{enc\PYZus{}layer\PYZus{}1}\PY{p}{,} \PY{l+m+mi}{128}\PY{p}{)}
            \PY{n}{dec\PYZus{}layer\PYZus{}2} \PY{o}{=} \PY{n}{decoder\PYZus{}block}\PY{p}{(}\PY{n}{dec\PYZus{}layer\PYZus{}1}\PY{p}{,} \PY{n}{inputs}\PY{p}{,} \PY{l+m+mi}{64}\PY{p}{)}
            
            \PY{c+c1}{\PYZsh{} The function returns the output layer of your model. \PYZdq{}x\PYZdq{} is the final layer obtained from the last decoder\PYZus{}block()}
            \PY{k}{return} \PY{n}{layers}\PY{o}{.}\PY{n}{Conv2D}\PY{p}{(}\PY{n}{num\PYZus{}classes}\PY{p}{,} \PY{l+m+mi}{1}\PY{p}{,} \PY{n}{activation}\PY{o}{=}\PY{l+s+s1}{\PYZsq{}}\PY{l+s+s1}{softmax}\PY{l+s+s1}{\PYZsq{}}\PY{p}{,} \PY{n}{padding}\PY{o}{=}\PY{l+s+s1}{\PYZsq{}}\PY{l+s+s1}{same}\PY{l+s+s1}{\PYZsq{}}\PY{p}{)}\PY{p}{(}\PY{n}{dec\PYZus{}layer\PYZus{}2}\PY{p}{)}
\end{Verbatim}


    \hypertarget{training}{%
\subsection{Training }\label{training}}

The following cells will use the FCN you created and define an ouput
layer based on the size of the processed image and the number of classes
recognized. You will define the hyperparameters to compile and train
your model.

Please Note: For this project, the helper code in
\texttt{data\_iterator.py} will resize the copter images to 160x160x3 to
speed up training.

    \begin{Verbatim}[commandchars=\\\{\}]
{\color{incolor}In [{\color{incolor}7}]:} \PY{l+s+sd}{\PYZdq{}\PYZdq{}\PYZdq{}}
        \PY{l+s+sd}{DON\PYZsq{}T MODIFY ANYTHING IN THIS CELL THAT IS BELOW THIS LINE}
        \PY{l+s+sd}{\PYZdq{}\PYZdq{}\PYZdq{}}
        
        \PY{n}{image\PYZus{}hw} \PY{o}{=} \PY{l+m+mi}{160}
        \PY{n}{image\PYZus{}shape} \PY{o}{=} \PY{p}{(}\PY{n}{image\PYZus{}hw}\PY{p}{,} \PY{n}{image\PYZus{}hw}\PY{p}{,} \PY{l+m+mi}{3}\PY{p}{)}
        \PY{n}{inputs} \PY{o}{=} \PY{n}{layers}\PY{o}{.}\PY{n}{Input}\PY{p}{(}\PY{n}{image\PYZus{}shape}\PY{p}{)}
        \PY{n}{num\PYZus{}classes} \PY{o}{=} \PY{l+m+mi}{3}
        
        \PY{c+c1}{\PYZsh{} Call fcn\PYZus{}model()}
        \PY{n}{output\PYZus{}layer} \PY{o}{=} \PY{n}{fcn\PYZus{}model}\PY{p}{(}\PY{n}{inputs}\PY{p}{,} \PY{n}{num\PYZus{}classes}\PY{p}{)}
\end{Verbatim}


    \hypertarget{hyperparameters}{%
\subsubsection{Hyperparameters}\label{hyperparameters}}

Define and tune your hyperparameters. - \textbf{batch\_size}: number of
training samples/images that get propagated through the network in a
single pass. - \textbf{num\_epochs}: number of times the entire training
dataset gets propagated through the network. -
\textbf{steps\_per\_epoch}: number of batches of training images that go
through the network in 1 epoch. We have provided you with a default
value. One recommended value to try would be based on the total number
of images in training dataset divided by the batch\_size. -
\textbf{validation\_steps}: number of batches of validation images that
go through the network in 1 epoch. This is similar to steps\_per\_epoch,
except validation\_steps is for the validation dataset. We have provided
you with a default value for this as well. - \textbf{workers}: maximum
number of processes to spin up. This can affect your training speed and
is dependent on your hardware. We have provided a recommended value to
work with.

    \begin{Verbatim}[commandchars=\\\{\}]
{\color{incolor}In [{\color{incolor}15}]:} \PY{n}{train\PYZus{}num} \PY{o}{=} \PY{l+m+mi}{4131}
         \PY{n}{val\PYZus{}num} \PY{o}{=} \PY{l+m+mi}{1184}
         
         \PY{n}{learning\PYZus{}rate} \PY{o}{=} \PY{l+m+mf}{0.001}
         \PY{n}{batch\PYZus{}size} \PY{o}{=} \PY{l+m+mi}{16}
         \PY{n}{num\PYZus{}epochs} \PY{o}{=} \PY{l+m+mi}{50}
         
         \PY{n}{steps\PYZus{}per\PYZus{}epoch} \PY{o}{=} \PY{n}{train\PYZus{}num} \PY{o}{/} \PY{n}{batch\PYZus{}size}
         \PY{n}{validation\PYZus{}steps} \PY{o}{=} \PY{n}{val\PYZus{}num} \PY{o}{/} \PY{n}{batch\PYZus{}size}
         
         \PY{n}{workers} \PY{o}{=} \PY{l+m+mi}{2}
\end{Verbatim}


    \begin{Verbatim}[commandchars=\\\{\}]
{\color{incolor}In [{\color{incolor}16}]:} \PY{l+s+sd}{\PYZdq{}\PYZdq{}\PYZdq{}}
         \PY{l+s+sd}{DON\PYZsq{}T MODIFY ANYTHING IN THIS CELL THAT IS BELOW THIS LINE}
         \PY{l+s+sd}{\PYZdq{}\PYZdq{}\PYZdq{}}
         \PY{c+c1}{\PYZsh{} Define the Keras model and compile it for training}
         \PY{n}{model} \PY{o}{=} \PY{n}{models}\PY{o}{.}\PY{n}{Model}\PY{p}{(}\PY{n}{inputs}\PY{o}{=}\PY{n}{inputs}\PY{p}{,} \PY{n}{outputs}\PY{o}{=}\PY{n}{output\PYZus{}layer}\PY{p}{)}
         
         \PY{n}{model}\PY{o}{.}\PY{n}{compile}\PY{p}{(}\PY{n}{optimizer}\PY{o}{=}\PY{n}{keras}\PY{o}{.}\PY{n}{optimizers}\PY{o}{.}\PY{n}{Adam}\PY{p}{(}\PY{n}{learning\PYZus{}rate}\PY{p}{)}\PY{p}{,} \PY{n}{loss}\PY{o}{=}\PY{l+s+s1}{\PYZsq{}}\PY{l+s+s1}{categorical\PYZus{}crossentropy}\PY{l+s+s1}{\PYZsq{}}\PY{p}{)}
         
         \PY{c+c1}{\PYZsh{} Data iterators for loading the training and validation data}
         \PY{n}{train\PYZus{}iter} \PY{o}{=} \PY{n}{data\PYZus{}iterator}\PY{o}{.}\PY{n}{BatchIteratorSimple}\PY{p}{(}\PY{n}{batch\PYZus{}size}\PY{o}{=}\PY{n}{batch\PYZus{}size}\PY{p}{,}
                                                        \PY{n}{data\PYZus{}folder}\PY{o}{=}\PY{n}{os}\PY{o}{.}\PY{n}{path}\PY{o}{.}\PY{n}{join}\PY{p}{(}\PY{l+s+s1}{\PYZsq{}}\PY{l+s+s1}{..}\PY{l+s+s1}{\PYZsq{}}\PY{p}{,} \PY{l+s+s1}{\PYZsq{}}\PY{l+s+s1}{data}\PY{l+s+s1}{\PYZsq{}}\PY{p}{,} \PY{l+s+s1}{\PYZsq{}}\PY{l+s+s1}{train}\PY{l+s+s1}{\PYZsq{}}\PY{p}{)}\PY{p}{,}
                                                        \PY{n}{image\PYZus{}shape}\PY{o}{=}\PY{n}{image\PYZus{}shape}\PY{p}{,}
                                                        \PY{n}{shift\PYZus{}aug}\PY{o}{=}\PY{k+kc}{True}\PY{p}{)}
         
         \PY{n}{val\PYZus{}iter} \PY{o}{=} \PY{n}{data\PYZus{}iterator}\PY{o}{.}\PY{n}{BatchIteratorSimple}\PY{p}{(}\PY{n}{batch\PYZus{}size}\PY{o}{=}\PY{n}{batch\PYZus{}size}\PY{p}{,}
                                                      \PY{n}{data\PYZus{}folder}\PY{o}{=}\PY{n}{os}\PY{o}{.}\PY{n}{path}\PY{o}{.}\PY{n}{join}\PY{p}{(}\PY{l+s+s1}{\PYZsq{}}\PY{l+s+s1}{..}\PY{l+s+s1}{\PYZsq{}}\PY{p}{,} \PY{l+s+s1}{\PYZsq{}}\PY{l+s+s1}{data}\PY{l+s+s1}{\PYZsq{}}\PY{p}{,} \PY{l+s+s1}{\PYZsq{}}\PY{l+s+s1}{validation}\PY{l+s+s1}{\PYZsq{}}\PY{p}{)}\PY{p}{,}
                                                      \PY{n}{image\PYZus{}shape}\PY{o}{=}\PY{n}{image\PYZus{}shape}\PY{p}{)}
         
         \PY{n}{logger\PYZus{}cb} \PY{o}{=} \PY{n}{plotting\PYZus{}tools}\PY{o}{.}\PY{n}{LoggerPlotter}\PY{p}{(}\PY{p}{)}
         \PY{n}{callbacks} \PY{o}{=} \PY{p}{[}\PY{n}{logger\PYZus{}cb}\PY{p}{]}
         
         \PY{n}{model}\PY{o}{.}\PY{n}{fit\PYZus{}generator}\PY{p}{(}\PY{n}{train\PYZus{}iter}\PY{p}{,}
                             \PY{n}{steps\PYZus{}per\PYZus{}epoch} \PY{o}{=} \PY{n}{steps\PYZus{}per\PYZus{}epoch}\PY{p}{,} \PY{c+c1}{\PYZsh{} the number of batches per epoch,}
                             \PY{n}{epochs} \PY{o}{=} \PY{n}{num\PYZus{}epochs}\PY{p}{,} \PY{c+c1}{\PYZsh{} the number of epochs to train for,}
                             \PY{n}{validation\PYZus{}data} \PY{o}{=} \PY{n}{val\PYZus{}iter}\PY{p}{,} \PY{c+c1}{\PYZsh{} validation iterator}
                             \PY{n}{validation\PYZus{}steps} \PY{o}{=} \PY{n}{validation\PYZus{}steps}\PY{p}{,} \PY{c+c1}{\PYZsh{} the number of batches to validate on}
                             \PY{n}{callbacks}\PY{o}{=}\PY{n}{callbacks}\PY{p}{,}
                             \PY{n}{workers} \PY{o}{=} \PY{n}{workers}\PY{p}{)}
\end{Verbatim}


    \begin{Verbatim}[commandchars=\\\{\}]
Epoch 1/50
258/258 [============================>.] - ETA: 0s - loss: 0.0286
    \end{Verbatim}

    \begin{center}
    \adjustimage{max size={0.9\linewidth}{0.9\paperheight}}{output_19_1.png}
    \end{center}
    { \hspace*{\fill} \\}
    
    \begin{Verbatim}[commandchars=\\\{\}]
259/258 [==============================] - 77s - loss: 0.0286 - val\_loss: 0.0365
Epoch 2/50
258/258 [============================>.] - ETA: 0s - loss: 0.0281
    \end{Verbatim}

    \begin{center}
    \adjustimage{max size={0.9\linewidth}{0.9\paperheight}}{output_19_3.png}
    \end{center}
    { \hspace*{\fill} \\}
    
    \begin{Verbatim}[commandchars=\\\{\}]
259/258 [==============================] - 78s - loss: 0.0280 - val\_loss: 0.0364
Epoch 3/50
258/258 [============================>.] - ETA: 0s - loss: 0.0276
    \end{Verbatim}

    \begin{center}
    \adjustimage{max size={0.9\linewidth}{0.9\paperheight}}{output_19_5.png}
    \end{center}
    { \hspace*{\fill} \\}
    
    \begin{Verbatim}[commandchars=\\\{\}]
259/258 [==============================] - 78s - loss: 0.0276 - val\_loss: 0.0360
Epoch 4/50
258/258 [============================>.] - ETA: 0s - loss: 0.0268
    \end{Verbatim}

    \begin{center}
    \adjustimage{max size={0.9\linewidth}{0.9\paperheight}}{output_19_7.png}
    \end{center}
    { \hspace*{\fill} \\}
    
    \begin{Verbatim}[commandchars=\\\{\}]
259/258 [==============================] - 78s - loss: 0.0268 - val\_loss: 0.0333
Epoch 5/50
258/258 [============================>.] - ETA: 0s - loss: 0.0264
    \end{Verbatim}

    \begin{center}
    \adjustimage{max size={0.9\linewidth}{0.9\paperheight}}{output_19_9.png}
    \end{center}
    { \hspace*{\fill} \\}
    
    \begin{Verbatim}[commandchars=\\\{\}]
259/258 [==============================] - 77s - loss: 0.0264 - val\_loss: 0.0423
Epoch 6/50
258/258 [============================>.] - ETA: 0s - loss: 0.0259
    \end{Verbatim}

    \begin{center}
    \adjustimage{max size={0.9\linewidth}{0.9\paperheight}}{output_19_11.png}
    \end{center}
    { \hspace*{\fill} \\}
    
    \begin{Verbatim}[commandchars=\\\{\}]
259/258 [==============================] - 76s - loss: 0.0259 - val\_loss: 0.0364
Epoch 7/50
258/258 [============================>.] - ETA: 0s - loss: 0.0255
    \end{Verbatim}

    \begin{center}
    \adjustimage{max size={0.9\linewidth}{0.9\paperheight}}{output_19_13.png}
    \end{center}
    { \hspace*{\fill} \\}
    
    \begin{Verbatim}[commandchars=\\\{\}]
259/258 [==============================] - 77s - loss: 0.0256 - val\_loss: 0.0290
Epoch 8/50
258/258 [============================>.] - ETA: 0s - loss: 0.0255
    \end{Verbatim}

    \begin{center}
    \adjustimage{max size={0.9\linewidth}{0.9\paperheight}}{output_19_15.png}
    \end{center}
    { \hspace*{\fill} \\}
    
    \begin{Verbatim}[commandchars=\\\{\}]
259/258 [==============================] - 76s - loss: 0.0254 - val\_loss: 0.0345
Epoch 9/50
258/258 [============================>.] - ETA: 0s - loss: 0.0247
    \end{Verbatim}

    \begin{center}
    \adjustimage{max size={0.9\linewidth}{0.9\paperheight}}{output_19_17.png}
    \end{center}
    { \hspace*{\fill} \\}
    
    \begin{Verbatim}[commandchars=\\\{\}]
259/258 [==============================] - 77s - loss: 0.0247 - val\_loss: 0.0264
Epoch 10/50
258/258 [============================>.] - ETA: 0s - loss: 0.0247
    \end{Verbatim}

    \begin{center}
    \adjustimage{max size={0.9\linewidth}{0.9\paperheight}}{output_19_19.png}
    \end{center}
    { \hspace*{\fill} \\}
    
    \begin{Verbatim}[commandchars=\\\{\}]
259/258 [==============================] - 76s - loss: 0.0247 - val\_loss: 0.0330
Epoch 11/50
258/258 [============================>.] - ETA: 0s - loss: 0.0241
    \end{Verbatim}

    \begin{center}
    \adjustimage{max size={0.9\linewidth}{0.9\paperheight}}{output_19_21.png}
    \end{center}
    { \hspace*{\fill} \\}
    
    \begin{Verbatim}[commandchars=\\\{\}]
259/258 [==============================] - 76s - loss: 0.0241 - val\_loss: 0.0370
Epoch 12/50
258/258 [============================>.] - ETA: 0s - loss: 0.0237
    \end{Verbatim}

    \begin{center}
    \adjustimage{max size={0.9\linewidth}{0.9\paperheight}}{output_19_23.png}
    \end{center}
    { \hspace*{\fill} \\}
    
    \begin{Verbatim}[commandchars=\\\{\}]
259/258 [==============================] - 76s - loss: 0.0237 - val\_loss: 0.0265
Epoch 13/50
258/258 [============================>.] - ETA: 0s - loss: 0.0236
    \end{Verbatim}

    \begin{center}
    \adjustimage{max size={0.9\linewidth}{0.9\paperheight}}{output_19_25.png}
    \end{center}
    { \hspace*{\fill} \\}
    
    \begin{Verbatim}[commandchars=\\\{\}]
259/258 [==============================] - 77s - loss: 0.0235 - val\_loss: 0.0319
Epoch 14/50
258/258 [============================>.] - ETA: 0s - loss: 0.0233
    \end{Verbatim}

    \begin{center}
    \adjustimage{max size={0.9\linewidth}{0.9\paperheight}}{output_19_27.png}
    \end{center}
    { \hspace*{\fill} \\}
    
    \begin{Verbatim}[commandchars=\\\{\}]
259/258 [==============================] - 76s - loss: 0.0233 - val\_loss: 0.0320
Epoch 15/50
258/258 [============================>.] - ETA: 0s - loss: 0.0232
    \end{Verbatim}

    \begin{center}
    \adjustimage{max size={0.9\linewidth}{0.9\paperheight}}{output_19_29.png}
    \end{center}
    { \hspace*{\fill} \\}
    
    \begin{Verbatim}[commandchars=\\\{\}]
259/258 [==============================] - 76s - loss: 0.0232 - val\_loss: 0.0301
Epoch 16/50
258/258 [============================>.] - ETA: 0s - loss: 0.0222
    \end{Verbatim}

    \begin{center}
    \adjustimage{max size={0.9\linewidth}{0.9\paperheight}}{output_19_31.png}
    \end{center}
    { \hspace*{\fill} \\}
    
    \begin{Verbatim}[commandchars=\\\{\}]
259/258 [==============================] - 76s - loss: 0.0227 - val\_loss: 0.0313
Epoch 17/50
258/258 [============================>.] - ETA: 0s - loss: 0.0227
    \end{Verbatim}

    \begin{center}
    \adjustimage{max size={0.9\linewidth}{0.9\paperheight}}{output_19_33.png}
    \end{center}
    { \hspace*{\fill} \\}
    
    \begin{Verbatim}[commandchars=\\\{\}]
259/258 [==============================] - 76s - loss: 0.0226 - val\_loss: 0.0294
Epoch 18/50
258/258 [============================>.] - ETA: 0s - loss: 0.0225
    \end{Verbatim}

    \begin{center}
    \adjustimage{max size={0.9\linewidth}{0.9\paperheight}}{output_19_35.png}
    \end{center}
    { \hspace*{\fill} \\}
    
    \begin{Verbatim}[commandchars=\\\{\}]
259/258 [==============================] - 77s - loss: 0.0225 - val\_loss: 0.0287
Epoch 19/50
258/258 [============================>.] - ETA: 0s - loss: 0.0222
    \end{Verbatim}

    \begin{center}
    \adjustimage{max size={0.9\linewidth}{0.9\paperheight}}{output_19_37.png}
    \end{center}
    { \hspace*{\fill} \\}
    
    \begin{Verbatim}[commandchars=\\\{\}]
259/258 [==============================] - 77s - loss: 0.0222 - val\_loss: 0.0355
Epoch 20/50
258/258 [============================>.] - ETA: 0s - loss: 0.0221
    \end{Verbatim}

    \begin{center}
    \adjustimage{max size={0.9\linewidth}{0.9\paperheight}}{output_19_39.png}
    \end{center}
    { \hspace*{\fill} \\}
    
    \begin{Verbatim}[commandchars=\\\{\}]
259/258 [==============================] - 77s - loss: 0.0221 - val\_loss: 0.0300
Epoch 21/50
258/258 [============================>.] - ETA: 0s - loss: 0.0217
    \end{Verbatim}

    \begin{center}
    \adjustimage{max size={0.9\linewidth}{0.9\paperheight}}{output_19_41.png}
    \end{center}
    { \hspace*{\fill} \\}
    
    \begin{Verbatim}[commandchars=\\\{\}]
259/258 [==============================] - 77s - loss: 0.0218 - val\_loss: 0.0301
Epoch 22/50
258/258 [============================>.] - ETA: 0s - loss: 0.0215
    \end{Verbatim}

    \begin{center}
    \adjustimage{max size={0.9\linewidth}{0.9\paperheight}}{output_19_43.png}
    \end{center}
    { \hspace*{\fill} \\}
    
    \begin{Verbatim}[commandchars=\\\{\}]
259/258 [==============================] - 76s - loss: 0.0215 - val\_loss: 0.0299
Epoch 23/50
258/258 [============================>.] - ETA: 0s - loss: 0.0213
    \end{Verbatim}

    \begin{center}
    \adjustimage{max size={0.9\linewidth}{0.9\paperheight}}{output_19_45.png}
    \end{center}
    { \hspace*{\fill} \\}
    
    \begin{Verbatim}[commandchars=\\\{\}]
259/258 [==============================] - 77s - loss: 0.0213 - val\_loss: 0.0353
Epoch 24/50
258/258 [============================>.] - ETA: 0s - loss: 0.0214
    \end{Verbatim}

    \begin{center}
    \adjustimage{max size={0.9\linewidth}{0.9\paperheight}}{output_19_47.png}
    \end{center}
    { \hspace*{\fill} \\}
    
    \begin{Verbatim}[commandchars=\\\{\}]
259/258 [==============================] - 77s - loss: 0.0214 - val\_loss: 0.0252
Epoch 25/50
258/258 [============================>.] - ETA: 0s - loss: 0.0210
    \end{Verbatim}

    \begin{center}
    \adjustimage{max size={0.9\linewidth}{0.9\paperheight}}{output_19_49.png}
    \end{center}
    { \hspace*{\fill} \\}
    
    \begin{Verbatim}[commandchars=\\\{\}]
259/258 [==============================] - 77s - loss: 0.0210 - val\_loss: 0.0356
Epoch 26/50
258/258 [============================>.] - ETA: 0s - loss: 0.0212
    \end{Verbatim}

    \begin{center}
    \adjustimage{max size={0.9\linewidth}{0.9\paperheight}}{output_19_51.png}
    \end{center}
    { \hspace*{\fill} \\}
    
    \begin{Verbatim}[commandchars=\\\{\}]
259/258 [==============================] - 77s - loss: 0.0212 - val\_loss: 0.0287
Epoch 27/50
258/258 [============================>.] - ETA: 0s - loss: 0.0210
    \end{Verbatim}

    \begin{center}
    \adjustimage{max size={0.9\linewidth}{0.9\paperheight}}{output_19_53.png}
    \end{center}
    { \hspace*{\fill} \\}
    
    \begin{Verbatim}[commandchars=\\\{\}]
259/258 [==============================] - 77s - loss: 0.0210 - val\_loss: 0.0292
Epoch 28/50
258/258 [============================>.] - ETA: 0s - loss: 0.0207
    \end{Verbatim}

    \begin{center}
    \adjustimage{max size={0.9\linewidth}{0.9\paperheight}}{output_19_55.png}
    \end{center}
    { \hspace*{\fill} \\}
    
    \begin{Verbatim}[commandchars=\\\{\}]
259/258 [==============================] - 77s - loss: 0.0207 - val\_loss: 0.0291
Epoch 29/50
258/258 [============================>.] - ETA: 0s - loss: 0.0204
    \end{Verbatim}

    \begin{center}
    \adjustimage{max size={0.9\linewidth}{0.9\paperheight}}{output_19_57.png}
    \end{center}
    { \hspace*{\fill} \\}
    
    \begin{Verbatim}[commandchars=\\\{\}]
259/258 [==============================] - 76s - loss: 0.0204 - val\_loss: 0.0289
Epoch 30/50
258/258 [============================>.] - ETA: 0s - loss: 0.0208
    \end{Verbatim}

    \begin{center}
    \adjustimage{max size={0.9\linewidth}{0.9\paperheight}}{output_19_59.png}
    \end{center}
    { \hspace*{\fill} \\}
    
    \begin{Verbatim}[commandchars=\\\{\}]
259/258 [==============================] - 76s - loss: 0.0208 - val\_loss: 0.0215
Epoch 31/50
258/258 [============================>.] - ETA: 0s - loss: 0.0204
    \end{Verbatim}

    \begin{center}
    \adjustimage{max size={0.9\linewidth}{0.9\paperheight}}{output_19_61.png}
    \end{center}
    { \hspace*{\fill} \\}
    
    \begin{Verbatim}[commandchars=\\\{\}]
259/258 [==============================] - 76s - loss: 0.0204 - val\_loss: 0.0340
Epoch 32/50
258/258 [============================>.] - ETA: 0s - loss: 0.0204
    \end{Verbatim}

    \begin{center}
    \adjustimage{max size={0.9\linewidth}{0.9\paperheight}}{output_19_63.png}
    \end{center}
    { \hspace*{\fill} \\}
    
    \begin{Verbatim}[commandchars=\\\{\}]
259/258 [==============================] - 76s - loss: 0.0204 - val\_loss: 0.0290
Epoch 33/50
258/258 [============================>.] - ETA: 0s - loss: 0.0201
    \end{Verbatim}

    \begin{center}
    \adjustimage{max size={0.9\linewidth}{0.9\paperheight}}{output_19_65.png}
    \end{center}
    { \hspace*{\fill} \\}
    
    \begin{Verbatim}[commandchars=\\\{\}]
259/258 [==============================] - 76s - loss: 0.0201 - val\_loss: 0.0269
Epoch 34/50
258/258 [============================>.] - ETA: 0s - loss: 0.0203
    \end{Verbatim}

    \begin{center}
    \adjustimage{max size={0.9\linewidth}{0.9\paperheight}}{output_19_67.png}
    \end{center}
    { \hspace*{\fill} \\}
    
    \begin{Verbatim}[commandchars=\\\{\}]
259/258 [==============================] - 76s - loss: 0.0203 - val\_loss: 0.0297
Epoch 35/50
258/258 [============================>.] - ETA: 0s - loss: 0.0200
    \end{Verbatim}

    \begin{center}
    \adjustimage{max size={0.9\linewidth}{0.9\paperheight}}{output_19_69.png}
    \end{center}
    { \hspace*{\fill} \\}
    
    \begin{Verbatim}[commandchars=\\\{\}]
259/258 [==============================] - 76s - loss: 0.0200 - val\_loss: 0.0281
Epoch 36/50
258/258 [============================>.] - ETA: 0s - loss: 0.0199
    \end{Verbatim}

    \begin{center}
    \adjustimage{max size={0.9\linewidth}{0.9\paperheight}}{output_19_71.png}
    \end{center}
    { \hspace*{\fill} \\}
    
    \begin{Verbatim}[commandchars=\\\{\}]
259/258 [==============================] - 76s - loss: 0.0198 - val\_loss: 0.0311
Epoch 37/50
258/258 [============================>.] - ETA: 0s - loss: 0.0198
    \end{Verbatim}

    \begin{center}
    \adjustimage{max size={0.9\linewidth}{0.9\paperheight}}{output_19_73.png}
    \end{center}
    { \hspace*{\fill} \\}
    
    \begin{Verbatim}[commandchars=\\\{\}]
259/258 [==============================] - 76s - loss: 0.0198 - val\_loss: 0.0308
Epoch 38/50
258/258 [============================>.] - ETA: 0s - loss: 0.0198
    \end{Verbatim}

    \begin{center}
    \adjustimage{max size={0.9\linewidth}{0.9\paperheight}}{output_19_75.png}
    \end{center}
    { \hspace*{\fill} \\}
    
    \begin{Verbatim}[commandchars=\\\{\}]
259/258 [==============================] - 76s - loss: 0.0198 - val\_loss: 0.0298
Epoch 39/50
258/258 [============================>.] - ETA: 0s - loss: 0.0194
    \end{Verbatim}

    \begin{center}
    \adjustimage{max size={0.9\linewidth}{0.9\paperheight}}{output_19_77.png}
    \end{center}
    { \hspace*{\fill} \\}
    
    \begin{Verbatim}[commandchars=\\\{\}]
259/258 [==============================] - 76s - loss: 0.0194 - val\_loss: 0.0263
Epoch 40/50
258/258 [============================>.] - ETA: 0s - loss: 0.0194
    \end{Verbatim}

    \begin{center}
    \adjustimage{max size={0.9\linewidth}{0.9\paperheight}}{output_19_79.png}
    \end{center}
    { \hspace*{\fill} \\}
    
    \begin{Verbatim}[commandchars=\\\{\}]
259/258 [==============================] - 76s - loss: 0.0193 - val\_loss: 0.0346
Epoch 41/50
258/258 [============================>.] - ETA: 0s - loss: 0.0191
    \end{Verbatim}

    \begin{center}
    \adjustimage{max size={0.9\linewidth}{0.9\paperheight}}{output_19_81.png}
    \end{center}
    { \hspace*{\fill} \\}
    
    \begin{Verbatim}[commandchars=\\\{\}]
259/258 [==============================] - 76s - loss: 0.0191 - val\_loss: 0.0233
Epoch 42/50
258/258 [============================>.] - ETA: 0s - loss: 0.0192
    \end{Verbatim}

    \begin{center}
    \adjustimage{max size={0.9\linewidth}{0.9\paperheight}}{output_19_83.png}
    \end{center}
    { \hspace*{\fill} \\}
    
    \begin{Verbatim}[commandchars=\\\{\}]
259/258 [==============================] - 77s - loss: 0.0192 - val\_loss: 0.0336
Epoch 43/50
258/258 [============================>.] - ETA: 0s - loss: 0.0194
    \end{Verbatim}

    \begin{center}
    \adjustimage{max size={0.9\linewidth}{0.9\paperheight}}{output_19_85.png}
    \end{center}
    { \hspace*{\fill} \\}
    
    \begin{Verbatim}[commandchars=\\\{\}]
259/258 [==============================] - 76s - loss: 0.0194 - val\_loss: 0.0285
Epoch 44/50
258/258 [============================>.] - ETA: 0s - loss: 0.0191
    \end{Verbatim}

    \begin{center}
    \adjustimage{max size={0.9\linewidth}{0.9\paperheight}}{output_19_87.png}
    \end{center}
    { \hspace*{\fill} \\}
    
    \begin{Verbatim}[commandchars=\\\{\}]
259/258 [==============================] - 76s - loss: 0.0191 - val\_loss: 0.0254
Epoch 45/50
258/258 [============================>.] - ETA: 0s - loss: 0.0189
    \end{Verbatim}

    \begin{center}
    \adjustimage{max size={0.9\linewidth}{0.9\paperheight}}{output_19_89.png}
    \end{center}
    { \hspace*{\fill} \\}
    
    \begin{Verbatim}[commandchars=\\\{\}]
259/258 [==============================] - 76s - loss: 0.0189 - val\_loss: 0.0340
Epoch 46/50
258/258 [============================>.] - ETA: 0s - loss: 0.0185
    \end{Verbatim}

    \begin{center}
    \adjustimage{max size={0.9\linewidth}{0.9\paperheight}}{output_19_91.png}
    \end{center}
    { \hspace*{\fill} \\}
    
    \begin{Verbatim}[commandchars=\\\{\}]
259/258 [==============================] - 76s - loss: 0.0185 - val\_loss: 0.0286
Epoch 47/50
258/258 [============================>.] - ETA: 0s - loss: 0.0190
    \end{Verbatim}

    \begin{center}
    \adjustimage{max size={0.9\linewidth}{0.9\paperheight}}{output_19_93.png}
    \end{center}
    { \hspace*{\fill} \\}
    
    \begin{Verbatim}[commandchars=\\\{\}]
259/258 [==============================] - 76s - loss: 0.0190 - val\_loss: 0.0263
Epoch 48/50
258/258 [============================>.] - ETA: 0s - loss: 0.0188
    \end{Verbatim}

    \begin{center}
    \adjustimage{max size={0.9\linewidth}{0.9\paperheight}}{output_19_95.png}
    \end{center}
    { \hspace*{\fill} \\}
    
    \begin{Verbatim}[commandchars=\\\{\}]
259/258 [==============================] - 76s - loss: 0.0188 - val\_loss: 0.0286
Epoch 49/50
258/258 [============================>.] - ETA: 0s - loss: 0.0186
    \end{Verbatim}

    \begin{center}
    \adjustimage{max size={0.9\linewidth}{0.9\paperheight}}{output_19_97.png}
    \end{center}
    { \hspace*{\fill} \\}
    
    \begin{Verbatim}[commandchars=\\\{\}]
259/258 [==============================] - 76s - loss: 0.0185 - val\_loss: 0.0280
Epoch 50/50
258/258 [============================>.] - ETA: 0s - loss: 0.0187
    \end{Verbatim}

    \begin{center}
    \adjustimage{max size={0.9\linewidth}{0.9\paperheight}}{output_19_99.png}
    \end{center}
    { \hspace*{\fill} \\}
    
    \begin{Verbatim}[commandchars=\\\{\}]
259/258 [==============================] - 76s - loss: 0.0187 - val\_loss: 0.0276

    \end{Verbatim}

\begin{Verbatim}[commandchars=\\\{\}]
{\color{outcolor}Out[{\color{outcolor}16}]:} <tensorflow.contrib.keras.python.keras.callbacks.History at 0x7fdfa84b2ba8>
\end{Verbatim}
            
    \begin{Verbatim}[commandchars=\\\{\}]
{\color{incolor}In [{\color{incolor}17}]:} \PY{c+c1}{\PYZsh{} Save your trained model weights}
         \PY{n}{weight\PYZus{}file\PYZus{}name} \PY{o}{=} \PY{l+s+s1}{\PYZsq{}}\PY{l+s+s1}{model\PYZus{}weights\PYZus{}3}\PY{l+s+s1}{\PYZsq{}}
         \PY{n}{model\PYZus{}tools}\PY{o}{.}\PY{n}{save\PYZus{}network}\PY{p}{(}\PY{n}{model}\PY{p}{,} \PY{n}{weight\PYZus{}file\PYZus{}name}\PY{p}{)}
\end{Verbatim}


    \hypertarget{prediction}{%
\subsection{Prediction }\label{prediction}}

Now that you have your model trained and saved, you can make predictions
on your validation dataset. These predictions can be compared to the
mask images, which are the ground truth labels, to evaluate how well
your model is doing under different conditions.

There are three different predictions available from the helper code
provided: - \textbf{patrol\_with\_targ}: Test how well the network can
detect the hero from a distance. - \textbf{patrol\_non\_targ}: Test how
often the network makes a mistake and identifies the wrong person as the
target. - \textbf{following\_images}: Test how well the network can
identify the target while following them.

    \begin{Verbatim}[commandchars=\\\{\}]
{\color{incolor}In [{\color{incolor}18}]:} \PY{c+c1}{\PYZsh{} If you need to load a model which you previously trained you can uncomment the codeline that calls the function below.}
         
         \PY{c+c1}{\PYZsh{} weight\PYZus{}file\PYZus{}name = \PYZsq{}model\PYZus{}weights\PYZsq{}}
         \PY{c+c1}{\PYZsh{} restored\PYZus{}model = model\PYZus{}tools.load\PYZus{}network(weight\PYZus{}file\PYZus{}name)}
\end{Verbatim}


    The following cell will write predictions to files and return paths to
the appropriate directories. The \texttt{run\_num} parameter is used to
define or group all the data for a particular model run. You can change
it for different runs. For example, `run\_1', `run\_2' etc.

    \begin{Verbatim}[commandchars=\\\{\}]
{\color{incolor}In [{\color{incolor}22}]:} \PY{n}{run\PYZus{}num} \PY{o}{=} \PY{l+s+s1}{\PYZsq{}}\PY{l+s+s1}{run\PYZus{}3}\PY{l+s+s1}{\PYZsq{}}
         
         \PY{n}{val\PYZus{}with\PYZus{}targ}\PY{p}{,} \PY{n}{pred\PYZus{}with\PYZus{}targ} \PY{o}{=} \PY{n}{model\PYZus{}tools}\PY{o}{.}\PY{n}{write\PYZus{}predictions\PYZus{}grade\PYZus{}set}\PY{p}{(}\PY{n}{model}\PY{p}{,}
                                                 \PY{n}{run\PYZus{}num}\PY{p}{,}\PY{l+s+s1}{\PYZsq{}}\PY{l+s+s1}{patrol\PYZus{}with\PYZus{}targ}\PY{l+s+s1}{\PYZsq{}}\PY{p}{,} \PY{l+s+s1}{\PYZsq{}}\PY{l+s+s1}{sample\PYZus{}evaluation\PYZus{}data}\PY{l+s+s1}{\PYZsq{}}\PY{p}{)} 
         
         \PY{n}{val\PYZus{}no\PYZus{}targ}\PY{p}{,} \PY{n}{pred\PYZus{}no\PYZus{}targ} \PY{o}{=} \PY{n}{model\PYZus{}tools}\PY{o}{.}\PY{n}{write\PYZus{}predictions\PYZus{}grade\PYZus{}set}\PY{p}{(}\PY{n}{model}\PY{p}{,} 
                                                 \PY{n}{run\PYZus{}num}\PY{p}{,}\PY{l+s+s1}{\PYZsq{}}\PY{l+s+s1}{patrol\PYZus{}non\PYZus{}targ}\PY{l+s+s1}{\PYZsq{}}\PY{p}{,} \PY{l+s+s1}{\PYZsq{}}\PY{l+s+s1}{sample\PYZus{}evaluation\PYZus{}data}\PY{l+s+s1}{\PYZsq{}}\PY{p}{)} 
         
         \PY{n}{val\PYZus{}following}\PY{p}{,} \PY{n}{pred\PYZus{}following} \PY{o}{=} \PY{n}{model\PYZus{}tools}\PY{o}{.}\PY{n}{write\PYZus{}predictions\PYZus{}grade\PYZus{}set}\PY{p}{(}\PY{n}{model}\PY{p}{,}
                                                 \PY{n}{run\PYZus{}num}\PY{p}{,}\PY{l+s+s1}{\PYZsq{}}\PY{l+s+s1}{following\PYZus{}images}\PY{l+s+s1}{\PYZsq{}}\PY{p}{,} \PY{l+s+s1}{\PYZsq{}}\PY{l+s+s1}{sample\PYZus{}evaluation\PYZus{}data}\PY{l+s+s1}{\PYZsq{}}\PY{p}{)}
\end{Verbatim}


    Now lets look at your predictions, and compare them to the ground truth
labels and original images. Run each of the following cells to visualize
some sample images from the predictions in the validation set.

    \begin{Verbatim}[commandchars=\\\{\}]
{\color{incolor}In [{\color{incolor}23}]:} \PY{c+c1}{\PYZsh{} images while following the target}
         \PY{n}{im\PYZus{}files} \PY{o}{=} \PY{n}{plotting\PYZus{}tools}\PY{o}{.}\PY{n}{get\PYZus{}im\PYZus{}file\PYZus{}sample}\PY{p}{(}\PY{l+s+s1}{\PYZsq{}}\PY{l+s+s1}{sample\PYZus{}evaluation\PYZus{}data}\PY{l+s+s1}{\PYZsq{}}\PY{p}{,}\PY{l+s+s1}{\PYZsq{}}\PY{l+s+s1}{following\PYZus{}images}\PY{l+s+s1}{\PYZsq{}}\PY{p}{,} \PY{n}{run\PYZus{}num}\PY{p}{)} 
         \PY{k}{for} \PY{n}{i} \PY{o+ow}{in} \PY{n+nb}{range}\PY{p}{(}\PY{l+m+mi}{3}\PY{p}{)}\PY{p}{:}
             \PY{n}{im\PYZus{}tuple} \PY{o}{=} \PY{n}{plotting\PYZus{}tools}\PY{o}{.}\PY{n}{load\PYZus{}images}\PY{p}{(}\PY{n}{im\PYZus{}files}\PY{p}{[}\PY{n}{i}\PY{p}{]}\PY{p}{)}
             \PY{n}{plotting\PYZus{}tools}\PY{o}{.}\PY{n}{show\PYZus{}images}\PY{p}{(}\PY{n}{im\PYZus{}tuple}\PY{p}{)}
             
\end{Verbatim}


    \begin{center}
    \adjustimage{max size={0.9\linewidth}{0.9\paperheight}}{output_26_0.png}
    \end{center}
    { \hspace*{\fill} \\}
    
    \begin{center}
    \adjustimage{max size={0.9\linewidth}{0.9\paperheight}}{output_26_1.png}
    \end{center}
    { \hspace*{\fill} \\}
    
    \begin{center}
    \adjustimage{max size={0.9\linewidth}{0.9\paperheight}}{output_26_2.png}
    \end{center}
    { \hspace*{\fill} \\}
    
    \begin{Verbatim}[commandchars=\\\{\}]
{\color{incolor}In [{\color{incolor}24}]:} \PY{c+c1}{\PYZsh{} images while at patrol without target}
         \PY{n}{im\PYZus{}files} \PY{o}{=} \PY{n}{plotting\PYZus{}tools}\PY{o}{.}\PY{n}{get\PYZus{}im\PYZus{}file\PYZus{}sample}\PY{p}{(}\PY{l+s+s1}{\PYZsq{}}\PY{l+s+s1}{sample\PYZus{}evaluation\PYZus{}data}\PY{l+s+s1}{\PYZsq{}}\PY{p}{,}\PY{l+s+s1}{\PYZsq{}}\PY{l+s+s1}{patrol\PYZus{}non\PYZus{}targ}\PY{l+s+s1}{\PYZsq{}}\PY{p}{,} \PY{n}{run\PYZus{}num}\PY{p}{)} 
         \PY{k}{for} \PY{n}{i} \PY{o+ow}{in} \PY{n+nb}{range}\PY{p}{(}\PY{l+m+mi}{3}\PY{p}{)}\PY{p}{:}
             \PY{n}{im\PYZus{}tuple} \PY{o}{=} \PY{n}{plotting\PYZus{}tools}\PY{o}{.}\PY{n}{load\PYZus{}images}\PY{p}{(}\PY{n}{im\PYZus{}files}\PY{p}{[}\PY{n}{i}\PY{p}{]}\PY{p}{)}
             \PY{n}{plotting\PYZus{}tools}\PY{o}{.}\PY{n}{show\PYZus{}images}\PY{p}{(}\PY{n}{im\PYZus{}tuple}\PY{p}{)}
          
\end{Verbatim}


    \begin{center}
    \adjustimage{max size={0.9\linewidth}{0.9\paperheight}}{output_27_0.png}
    \end{center}
    { \hspace*{\fill} \\}
    
    \begin{center}
    \adjustimage{max size={0.9\linewidth}{0.9\paperheight}}{output_27_1.png}
    \end{center}
    { \hspace*{\fill} \\}
    
    \begin{center}
    \adjustimage{max size={0.9\linewidth}{0.9\paperheight}}{output_27_2.png}
    \end{center}
    { \hspace*{\fill} \\}
    
    \begin{Verbatim}[commandchars=\\\{\}]
{\color{incolor}In [{\color{incolor}25}]:}    
         \PY{c+c1}{\PYZsh{} images while at patrol with target}
         \PY{n}{im\PYZus{}files} \PY{o}{=} \PY{n}{plotting\PYZus{}tools}\PY{o}{.}\PY{n}{get\PYZus{}im\PYZus{}file\PYZus{}sample}\PY{p}{(}\PY{l+s+s1}{\PYZsq{}}\PY{l+s+s1}{sample\PYZus{}evaluation\PYZus{}data}\PY{l+s+s1}{\PYZsq{}}\PY{p}{,}\PY{l+s+s1}{\PYZsq{}}\PY{l+s+s1}{patrol\PYZus{}with\PYZus{}targ}\PY{l+s+s1}{\PYZsq{}}\PY{p}{,} \PY{n}{run\PYZus{}num}\PY{p}{)} 
         \PY{k}{for} \PY{n}{i} \PY{o+ow}{in} \PY{n+nb}{range}\PY{p}{(}\PY{l+m+mi}{3}\PY{p}{)}\PY{p}{:}
             \PY{n}{im\PYZus{}tuple} \PY{o}{=} \PY{n}{plotting\PYZus{}tools}\PY{o}{.}\PY{n}{load\PYZus{}images}\PY{p}{(}\PY{n}{im\PYZus{}files}\PY{p}{[}\PY{n}{i}\PY{p}{]}\PY{p}{)}
             \PY{n}{plotting\PYZus{}tools}\PY{o}{.}\PY{n}{show\PYZus{}images}\PY{p}{(}\PY{n}{im\PYZus{}tuple}\PY{p}{)}
\end{Verbatim}


    \begin{center}
    \adjustimage{max size={0.9\linewidth}{0.9\paperheight}}{output_28_0.png}
    \end{center}
    { \hspace*{\fill} \\}
    
    \begin{center}
    \adjustimage{max size={0.9\linewidth}{0.9\paperheight}}{output_28_1.png}
    \end{center}
    { \hspace*{\fill} \\}
    
    \begin{center}
    \adjustimage{max size={0.9\linewidth}{0.9\paperheight}}{output_28_2.png}
    \end{center}
    { \hspace*{\fill} \\}
    
    \hypertarget{evaluation}{%
\subsection{Evaluation }\label{evaluation}}

Evaluate your model! The following cells include several different
scores to help you evaluate your model under the different conditions
discussed during the Prediction step.

    \begin{Verbatim}[commandchars=\\\{\}]
{\color{incolor}In [{\color{incolor}26}]:} \PY{c+c1}{\PYZsh{} Scores for while the quad is following behind the target. }
         \PY{n}{true\PYZus{}pos1}\PY{p}{,} \PY{n}{false\PYZus{}pos1}\PY{p}{,} \PY{n}{false\PYZus{}neg1}\PY{p}{,} \PY{n}{iou1} \PY{o}{=} \PY{n}{scoring\PYZus{}utils}\PY{o}{.}\PY{n}{score\PYZus{}run\PYZus{}iou}\PY{p}{(}\PY{n}{val\PYZus{}following}\PY{p}{,} \PY{n}{pred\PYZus{}following}\PY{p}{)}
\end{Verbatim}


    \begin{Verbatim}[commandchars=\\\{\}]
number of validation samples intersection over the union evaulated on 542
average intersection over union for background is 0.9952384765113984
average intersection over union for other people is 0.33065800635333636
average intersection over union for the hero is 0.8860995780679758
number true positives: 539, number false positives: 0, number false negatives: 0

    \end{Verbatim}

    \begin{Verbatim}[commandchars=\\\{\}]
{\color{incolor}In [{\color{incolor}27}]:} \PY{c+c1}{\PYZsh{} Scores for images while the quad is on patrol and the target is not visable}
         \PY{n}{true\PYZus{}pos2}\PY{p}{,} \PY{n}{false\PYZus{}pos2}\PY{p}{,} \PY{n}{false\PYZus{}neg2}\PY{p}{,} \PY{n}{iou2} \PY{o}{=} \PY{n}{scoring\PYZus{}utils}\PY{o}{.}\PY{n}{score\PYZus{}run\PYZus{}iou}\PY{p}{(}\PY{n}{val\PYZus{}no\PYZus{}targ}\PY{p}{,} \PY{n}{pred\PYZus{}no\PYZus{}targ}\PY{p}{)}
\end{Verbatim}


    \begin{Verbatim}[commandchars=\\\{\}]
number of validation samples intersection over the union evaulated on 270
average intersection over union for background is 0.9842759153815338
average intersection over union for other people is 0.6723475784926007
average intersection over union for the hero is 0.0
number true positives: 0, number false positives: 83, number false negatives: 0

    \end{Verbatim}

    \begin{Verbatim}[commandchars=\\\{\}]
{\color{incolor}In [{\color{incolor}28}]:} \PY{c+c1}{\PYZsh{} This score measures how well the neural network can detect the target from far away}
         \PY{n}{true\PYZus{}pos3}\PY{p}{,} \PY{n}{false\PYZus{}pos3}\PY{p}{,} \PY{n}{false\PYZus{}neg3}\PY{p}{,} \PY{n}{iou3} \PY{o}{=} \PY{n}{scoring\PYZus{}utils}\PY{o}{.}\PY{n}{score\PYZus{}run\PYZus{}iou}\PY{p}{(}\PY{n}{val\PYZus{}with\PYZus{}targ}\PY{p}{,} \PY{n}{pred\PYZus{}with\PYZus{}targ}\PY{p}{)}
\end{Verbatim}


    \begin{Verbatim}[commandchars=\\\{\}]
number of validation samples intersection over the union evaulated on 322
average intersection over union for background is 0.995973641847985
average intersection over union for other people is 0.42339571964747114
average intersection over union for the hero is 0.21172049193366338
number true positives: 147, number false positives: 3, number false negatives: 154

    \end{Verbatim}

    \begin{Verbatim}[commandchars=\\\{\}]
{\color{incolor}In [{\color{incolor}29}]:} \PY{c+c1}{\PYZsh{} Sum all the true positives, etc from the three datasets to get a weight for the score}
         \PY{n}{true\PYZus{}pos} \PY{o}{=} \PY{n}{true\PYZus{}pos1} \PY{o}{+} \PY{n}{true\PYZus{}pos2} \PY{o}{+} \PY{n}{true\PYZus{}pos3}
         \PY{n}{false\PYZus{}pos} \PY{o}{=} \PY{n}{false\PYZus{}pos1} \PY{o}{+} \PY{n}{false\PYZus{}pos2} \PY{o}{+} \PY{n}{false\PYZus{}pos3}
         \PY{n}{false\PYZus{}neg} \PY{o}{=} \PY{n}{false\PYZus{}neg1} \PY{o}{+} \PY{n}{false\PYZus{}neg2} \PY{o}{+} \PY{n}{false\PYZus{}neg3}
         
         \PY{n}{weight} \PY{o}{=} \PY{n}{true\PYZus{}pos}\PY{o}{/}\PY{p}{(}\PY{n}{true\PYZus{}pos}\PY{o}{+}\PY{n}{false\PYZus{}neg}\PY{o}{+}\PY{n}{false\PYZus{}pos}\PY{p}{)}
         \PY{n+nb}{print}\PY{p}{(}\PY{n}{weight}\PY{p}{)}
\end{Verbatim}


    \begin{Verbatim}[commandchars=\\\{\}]
0.7408207343412527

    \end{Verbatim}

    \begin{Verbatim}[commandchars=\\\{\}]
{\color{incolor}In [{\color{incolor}30}]:} \PY{c+c1}{\PYZsh{} The IoU for the dataset that never includes the hero is excluded from grading}
         \PY{n}{final\PYZus{}IoU} \PY{o}{=} \PY{p}{(}\PY{n}{iou1} \PY{o}{+} \PY{n}{iou3}\PY{p}{)}\PY{o}{/}\PY{l+m+mi}{2}
         \PY{n+nb}{print}\PY{p}{(}\PY{n}{final\PYZus{}IoU}\PY{p}{)}
\end{Verbatim}


    \begin{Verbatim}[commandchars=\\\{\}]
0.548910035001

    \end{Verbatim}

    \begin{Verbatim}[commandchars=\\\{\}]
{\color{incolor}In [{\color{incolor}31}]:} \PY{c+c1}{\PYZsh{} And the final grade score is }
         \PY{n}{final\PYZus{}score} \PY{o}{=} \PY{n}{final\PYZus{}IoU} \PY{o}{*} \PY{n}{weight}
         \PY{n+nb}{print}\PY{p}{(}\PY{n}{final\PYZus{}score}\PY{p}{)}
\end{Verbatim}


    \begin{Verbatim}[commandchars=\\\{\}]
0.406643935217

    \end{Verbatim}


    % Add a bibliography block to the postdoc
    
    
    
    \end{document}
